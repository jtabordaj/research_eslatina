\documentclass[12pt,a4paper]{article}

\usepackage{amsmath}				% For use of  a large range of formulas, commands, and symbols
\usepackage[utf8]{inputenc}			% Specifies the encoding. (Zeichenkodierung, bindet Sonderzeichen mit ein)
\usepackage[T1]{fontenc}			% Include the accented characters as individual glyphs(ö is one single glyph, not an o with an added accent)
\usepackage[spanish]{babel}			% Specifies the language of doucment (translates "Table of Contents"...)
\usepackage{graphicx}				% For including graphics
\graphicspath{{images}}
\usepackage{subfigure}				% For including "sub"figures
\usepackage{lscape}					% Rotates text within environment by 90 degrees.
\usepackage{pst-plot, pstricks}		% plot­ting of data (typ­i­cally from ex­ter­nal files), plotting lines, rectangles...
\usepackage{fancybox,amssymb,color}	% fancybox: frames, rotations - amssymb: extension for amsmath - color: set the font color, text background, or page
\usepackage{setspace}				% allows to specifiy the space between lines
\usepackage{booktabs} 				% For prettier tables
\onehalfspacing
\pagestyle{plain}
\usepackage{enumerate}
\usepackage[hidelinks]{hyperref}
\setlength{\parindent}{1em}
\setlength{\parskip}{1ex}
\usepackage{geometry}   		 % definition of the page layout
\geometry{
	a4paper,					 % format
	left=30mm,					 % left margin
	right=30mm,					 % right margin
	top=25mm,					 % distance upwards
	bottom=25mm,				 % distance downwards
	includehead,				 % distance from upwards till the page header
}
\usepackage{booktabs}
\usepackage{longtable}	
\usepackage{adjustbox}	
\usepackage{multirow}		
\usepackage{float}	


\begin{document}

\begin{center}
	\textbf{Diccionario de Datos} \\
	Ultima vez editado: \today
\end{center}
\vspace{10mm}

\textbf{Nivel de desagregación de los datos}
\begin{enumerate}
	\item A nivel municipal: Departamento - Municipio
\end{enumerate}

Las bases de datos tienen como enfoque el consumo residencial, por lo que se encuentran los Estratos del 1 al 6, junto con el total residencial. Para el caso de consumo no residencial, como puede ser la industria, el sector oficial y otros agentes, solo se considera el total. Cada columna tiene en su nombre el agente en cuestión y, separado por un "\_", la variable reportada.

\begin{center}
	\textbf{\textbf{Ejemplo}: \textit{Estrato1\_pcon }indica que esa columna reporta los valores de consumo promedio para los hogares de Estrato 1 en cada municipio} 
\end{center}

Las bases de datos se componen de variables únicas, separadas según la fuente de energía:

\section{Energia electrica}

\begin{enumerate}
	\item Suscriptores (\textbf{sus}): Variable que reporta el total de usuarios en cada municipio.
	\item Consumo (\textbf{tcon}): Variable que reporta el total de consumo en $Kw/H$ por cada municipio.
	\item Valor de consumo (\textbf{vcon}): Variable que reporta el valor en unidades monetarias de los $Kw/H$ consumidos por cada municipio.
	\item Factura promedio (\textbf{pfac}): Variable que reporta el valor promedio en unidades monetarias de las facturas de energía eléctrica en cada municipio.
	\item Consumo promedio (\textbf{pcon}): Variable que reporta el valor promedio en $Kw/H$ del consumo de energía eléctrica en cada municipio.
	\item Tarifa media (\textbf{ptar}): Variable que reporta el valor promedio en unidades monetarias de un $Kw/H$ en cada municipio.
	\item Total facturado (\textbf{fac}): Variable que reporta el total de unidades monetarias facturadas a los usuarios por el consumo de energía eléctrica en cada municipio.
\end{enumerate}

\section{Gas Natural}

\begin{enumerate}
	\item Suscriptores (\textbf{sus}): Variable que reporta el total de usuarios en cada municipio.
	\item Consumo (\textbf{tcon}): Variable que reporta el total de consumo metros cúbicos ($M^3$) por cada municipio.
	\item Valor de consumo (\textbf{vcon}): Variable que reporta el valor en unidades monetarias de los $M^3$ consumidos por cada municipio.
	\item Total facturado (\textbf{fac}): Variable que reporta el total de unidades monetarias facturadas a los usuarios por el consumo de gas natural en cada municipio.
\end{enumerate}


\section*{Información Adicional}

La diferencia entre total facturado (fac) y valor de consumo (vcon) reside en que el valor de consumo no considera subsidios o reducciones a la factura del usuario final, mientras que el total facturado reporta el recaudo de la entidad prestadora de servicio. La sustracción entre ambas variables puede indicar el valor subsidiado o descontado del usuario final.

Para 2021, los siguientes municipios no cuentan con información para energía eléctrica:

\begin{table}[h!]
	\centering
	\begin{tabular}{@{}ll@{}}
		\toprule
		\textbf{Departamento} & \textbf{Municipio} \\ \midrule
		CAUCA                 & TIMBIQUI           \\ \bottomrule
	\end{tabular}
\end{table}

Para 2022, los siguientes municipios no cuentan con información para energía eléctrica:

\begin{longtable}{@{}lc@{}}
	\toprule
	\multicolumn{1}{c}{\textbf{Departamento}} & \textbf{Municipio}  \\ \midrule
		ATLANTICO             & MALAMBO              \\
		ATLANTICO             & SABANAGRANDE         \\
		ATLANTICO             & BARANOA              \\
		ATLANTICO             & CAMPO DE LA CRUZ     \\
		ATLANTICO             & CANDELARIA           \\
		ATLANTICO             & GALAPA               \\
		ATLANTICO             & JUAN DE ACOSTA       \\
		ATLANTICO             & LURUACO              \\
		ATLANTICO             & PALMAR DE VARELA     \\
		ATLANTICO             & PIOJO                \\
		ATLANTICO             & POLONUEVO            \\
		ATLANTICO             & PONEDERA             \\
		ATLANTICO             & SABANALARGA          \\
		ATLANTICO             & SANTA LUCIA          \\
		ATLANTICO             & SANTO TOMAS          \\
		ATLANTICO             & SOLEDAD              \\
		ATLANTICO             & SUAN                 \\
		ATLANTICO             & TUBARA               \\
		ATLANTICO             & USIACURI             \\
		LA GUAJIRA            & ALBANIA              \\
		LA GUAJIRA            & BARRANCAS            \\
		LA GUAJIRA            & DIBULLA              \\
		LA GUAJIRA            & DISTRACCION          \\
		LA GUAJIRA            & EL MOLINO            \\
		LA GUAJIRA            & FONSECA              \\
		LA GUAJIRA            & HATONUEVO            \\
		LA GUAJIRA            & MAICAO               \\
		LA GUAJIRA            & MANAURE              \\
		LA GUAJIRA            & RIOHACHA             \\
		LA GUAJIRA            & SAN JUAN DEL CESAR   \\
		LA GUAJIRA            & URIBIA               \\
		LA GUAJIRA            & URUMITA              \\
		LA GUAJIRA            & VILLANUEVA           \\
		MAGDALENA             & ARACATACA            \\
		MAGDALENA             & CERRO DE SAN ANTONIO \\
		MAGDALENA             & CHIVOLO              \\
		MAGDALENA             & CIENAGA              \\
		MAGDALENA             & CONCORDIA            \\
		MAGDALENA             & EL PINON             \\
		MAGDALENA             & EL RETEN             \\
		MAGDALENA             & FUNDACION            \\
		MAGDALENA             & PEDRAZA              \\
		MAGDALENA             & PIVIJAY              \\
		MAGDALENA             & PLATO                \\
		MAGDALENA             & PUEBLOVIEJO          \\
		MAGDALENA             & REMOLINO             \\
		MAGDALENA             & SALAMINA             \\
		MAGDALENA             & SITIONUEVO           \\
		MAGDALENA             & TENERIFE             \\
		MAGDALENA             & ZAPAYAN              \\
		MAGDALENA             & ZONA BANANERA        \\
		CAUCA                 & TIMBIQUI             \\
		PUTUMAYO              & COLON                \\
		PUTUMAYO              & SAN FRANCISCO        \\
		PUTUMAYO              & SANTIAGO             \\
		PUTUMAYO              & SIBUNDOY             \\ \bottomrule
\end{longtable}

Para gas natural entre 2021 y 2022, hay municipios que no están presentes en ambas bases de datos:

\begin{longtable}{@{}lc@{}}
	\toprule
	\multicolumn{1}{c}{\textbf{Departamento}} & \textbf{Municipio}  \\ \midrule
		SANTANDER          & ARATOCA               \\
		SANTANDER          & CAPITANEJO            \\
		SANTANDER          & CHIMA                 \\
		SANTANDER          & CIMITARRA             \\
		SANTANDER          & CONFINES              \\
		SANTANDER          & CONTRATACION          \\
		SANTANDER          & COROMORO              \\
		SANTANDER          & EL GUACAMAYO          \\
		SANTANDER          & ENCINO                \\
		SANTANDER          & ENCISO                \\
		SANTANDER          & GUACA                 \\
		SANTANDER          & GUADALUPE             \\
		SANTANDER          & HATO                  \\
		SANTANDER          & LANDAZURI             \\
		SANTANDER          & LOS SANTOS            \\
		SANTANDER          & MOGOTES               \\
		SANTANDER          & OCAMONTE              \\
		SANTANDER          & OIBA                  \\
		SANTANDER          & PALMAR                \\
		SANTANDER          & SAN ANDRES            \\
		SANTANDER          & SAN JOAQUIN           \\
		SANTANDER          & SANTA BARBARA         \\
		SANTANDER          & SANTA HELENA DEL OPON \\
		SANTANDER          & SIMACOTA              \\
		CUNDINAMARCA       & CARMEN DE CARUPA      \\
		CUNDINAMARCA       & GUTIERREZ             \\
		CUNDINAMARCA       & SAN CAYETANO          \\
		TOLIMA             & ANZOATEGUI            \\
		TOLIMA             & ATACO                 \\
		TOLIMA             & PLANADAS              \\
		TOLIMA             & RONCESVALLES          \\
		TOLIMA             & ROVIRA                \\
		SUCRE              & GUARANDA              \\
		HUILA              & ACEVEDO               \\
		HUILA              & COLOMBIA              \\
		HUILA              & ELIAS                 \\
		HUILA              & IQUIRA                \\
		HUILA              & ISNOS                 \\
		HUILA              & LA ARGENTINA          \\
		HUILA              & NATAGA                \\
		HUILA              & OPORAPA               \\
		HUILA              & PALESTINA             \\
		HUILA              & SALADOBLANCO          \\
		HUILA              & SANTA MARIA           \\
		CHOCO              & EL CARMEN DE ATRATO   \\
		NORTE DE SANTANDER & ARBOLEDAS             \\
		NORTE DE SANTANDER & CACOTA                \\
		NORTE DE SANTANDER & CUCUTILLA             \\
		NORTE DE SANTANDER & SALAZAR               \\
		NORTE DE SANTANDER & SANTIAGO              \\
		CALDAS             & SAMANA                \\
		ANTIOQUIA          & ANGOSTURA             \\
		ANTIOQUIA          & ANORI                 \\
		ANTIOQUIA          & CAMPAMENTO            \\
		ANTIOQUIA          & EBEJICO               \\
		ANTIOQUIA          & NECHI                 \\
		BOYACA             & UMBITA                \\
		BOYACA             & GAMEZA                \\
		BOYACA             & MONGUA                \\
		BOYACA             & MONGUI                \\
		BOYACA             & TOPAGA                \\
		BOYACA             & TOCA                  \\
		CAUCA              & INZA                  \\
		CAUCA              & PAEZ                  \\
		CAQUETA            & EL PAUJIL             \\
		CAQUETA            & SAN JOSE DEL FRAGUA   \\ \bottomrule
\end{longtable}

Todos los datos provienen del \textit{Sistema Único de Información de Servicios Públicos Domiciliarios} para 2021 y 2022. Para mas información, ver: \textbf{\href{http://sui.superservicios.gov.co/Reportes-del-sector/Energia/Reportes-comerciales/Consolidado-Energia-por-Empresa-Departamento-y-Municipio}{SUI}}.

\vfill

\noindent Contacto: \href{mailto:jtabordaj@uninorte.edu.co}{jtabordaj@uninorte.edu.co} \\
Repositorio: \textbf{\href{https://github.com/jtabordaj/research_eslatina}{GitHub}}


\end{document}